\documentclass[crop,tikz]{standalone}
\usepackage{tikz}
\usepackage{ifthen}
\usepackage[scaled]{helvet}
\renewcommand*\familydefault{\sfdefault}

\begin{document}

\definecolor{yellow}{rgb}{1.0, 1.0, 0.45} % 255/255/115
\definecolor{dkyellow}{rgb}{0.9, 0.9, 0.0} % % 230/230/0

\definecolor{ltorange}{rgb}{1.0, 0.74, 0.41} % 255/188/105
\definecolor{orange}{rgb}{0.96, 0.50, 0.0} % 246/127/0

\definecolor{ltred}{rgb}{1.0, 0.25, 0.25} % 255/64/64
\definecolor{red}{rgb}{0.79, 0.00, 0.01} % 201/0/3

\definecolor{ltpurple}{rgb}{0.81, 0.57, 1.00} % 206/145/255
\definecolor{purple}{rgb}{0.38, 0.00, 0.68} % 97/1/175

\definecolor{ltblue}{rgb}{0.2, 0.73, 1.0} % 51/187/255
\definecolor{blue}{rgb}{0.12, 0.43, 0.59} % 30/110/150

\definecolor{ltltgreen}{rgb}{0.7, 1.00, 0.7} % 96/204/14
\definecolor{ltgreen}{rgb}{0.37, 0.80, 0.05} % 96/204/14
\definecolor{green}{rgb}{0.23, 0.49, 0.03} % 59/125/8
  
\definecolor{dkslate}{rgb}{0.18, 0.21, 0.28} % 47/53/72
\definecolor{mdslate}{rgb}{0.45, 0.50, 0.68} % 114/127/173
\definecolor{ltslate}{rgb}{0.85, 0.88, 0.95} % 216/225/229

\usetikzlibrary{positioning,arrows.meta,shapes,calc,patterns}

% Basic entities
\tikzstyle{arrow} = [-{Latex[length=9pt,width=6pt]}]
\tikzstyle{rarrow} = [{Latex[length=9pt,width=6pt]}-]
\tikzstyle{vertex}=[circle,inner sep=1.5pt, minimum size=2mm]

\tikzstyle{bc}=[color=purple, line width=2.0pt]
\tikzstyle{bc-label}=[text=purple]

\tikzstyle{material}=[color=black, line width=2.0pt]
\tikzstyle{material-label}=[text=black]

\tikzstyle{gdomain}=[fill=ltblue!25!white]

% Finite-element entities
\tikzstyle{point-label}=[]
\tikzstyle{fevertex}=[vertex, color=orange,fill=orange]
\tikzstyle{fevertex-label}=[point-label, text=orange]

\tikzstyle{cell}=[color=blue,line width=1.0pt]
\tikzstyle{cell-label}=[point-label, text=blue]


% Geometry entities
\tikzstyle{gvertex}=[vertex, color=orange,fill=orange]
\tikzstyle{gvertex-label}=[point-label, text=orange]

\tikzstyle{curve}=[line width=1.0pt, color=blue]
\tikzstyle{curve-dir}=[line width=1.0pt, arrow=>latex, color=blue]
\tikzstyle{curve-label}=[point-label, text=blue]


% Annotation
\tikzstyle{axes}=[line width=1.0pt, color=gray]
\tikzstyle{axes-label}=[font=\bfseries, color=gray]


\tikzstyle{roller}=[color=green,line width=1.0pt]
\tikzstyle{ground}=[fill,pattern color=green,pattern=north east lines,draw=none,anchor=north,minimum width=0.8cm,minimum height=0.3cm]



\foreach \figpart in {1,...,8}{

% -------------------------------------------------------------------------------------------------
\begin{tikzpicture}[scale=0.05]

\def\domainx{100}
\def\domainy{150}
\def\axesshift{0.15*\domainx}
\def\bndryshift{1.5*\axesshift}
\def\ticksize{0.25*\axesshift}
\def\dirsize{0.15*\domainx}

% Reference points
\coordinate (p1) at (-0.5*\domainx,-0.5*\domainy);
\coordinate (p2) at (+0.5*\domainx,-0.5*\domainy);
\coordinate (p3) at (+0.5*\domainx,+0.5*\domainy);
\coordinate (p4) at (-0.5*\domainx,+0.5*\domainy);
\coordinate (p5) at (0.0mm,-0.5*\domainy);
\coordinate (p6) at (0.0mm,+0.5*\domainy);

% Domain
\draw[curve,gdomain] (p1) -- (p2) -- (p3) -- (p4) -- cycle;
\draw[curve] (p5) -- (p6);

% -------------------------------------------------------------------------------------------------
% Domain
% -------------------------------------------------------------------------------------------------
\ifthenelse{\figpart=1}{

% Axes
\draw[axes,arrow] ($(p1)+(0,-\axesshift)$) to ($(p2)+(1.0*\axesshift,-\axesshift)$);
\node[axes-label,right] at ($(p2)+(1.0*\axesshift,-\axesshift)$) {x};
\foreach \ix in {0,1}{%
    \pgfmathsetmacro{\x}{-0.5*\domainx+\ix*\domainx}
    \draw[axes] ($(p1)+(\ix*\domainx,-\axesshift)$) -- ++(0,-\ticksize) node[below]{\x\,km};
}

\draw[axes,arrow] ($(p1)+(-\axesshift,0)$) to ($(p4)+(-\axesshift,+1.0*\axesshift)$);
\node[axes-label,above] at ($(p4)+(-\axesshift,1.0*\axesshift)$) {y};
\foreach \iy in {0,1}{%
    \pgfmathsetmacro{\y}{-0.5*\domainy+\iy*\domainy}
    \draw[axes] ($(p1)+(-\axesshift,\iy*\domainy)$) -- ++(-\ticksize,0) node[left]{\y\,km};
}

% Materials
\node[bc-label] at ($(p1)+(0.25*\domainx,0.5*\domainy)$) {elastic\_xneg};
\node[bc-label] at ($(p2)+(-0.25*\domainx,0.5*\domainy)$) {elastic\_xpos};

% Boundaries
\node[bc-label, anchor=east] (xneg-label) at ($(p1)+(-\bndryshift,0.5*\domainy)$) {boundary\_xneg};
\draw[bc,arrow] (xneg-label) to[out=10,in=-170,looseness=2] ($(p1)+(0,0.5*\domainy)$);

\node[bc-label, anchor=west] (xpos-label) at ($(p2)+(+\bndryshift,0.5*\domainy)$) {boundary\_xpos};
\draw[bc,arrow] (xpos-label) to[out=-170,in=+10,looseness=2] ($(p2)+(0,0.5*\domainy)$);

\node[bc-label, anchor=north] (yneg-label) at ($(p1)+(0.5*\domainx,-\bndryshift)$) {boundary\_yneg};
\draw[bc,arrow] (yneg-label) to[out=100,in=-80,looseness=2] ($(p1)+(0.5*\domainx,0)$);

\node[bc-label, anchor=south] (ypos-label) at ($(p4)+(0.5*\domainx,\bndryshift)$) {boundary\_ypos};
\draw[bc,arrow] (ypos-label) to[out=-80,in=100,looseness=2] ($(p4)+(0.5*\domainx,0)$);

\node[bc-label, anchor=west] (fault-label) at ($(p5)+(\bndryshift,0.25*\domainy)$) {fault};
\draw[bc,arrow] (fault-label) to[out=-170,in=+10,looseness=2] ($(p5)+(0,0.25*\domainy)$);

}{}


% -------------------------------------------------------------------------------------------------
% Gmsh Geometry
% -------------------------------------------------------------------------------------------------
\ifthenelse{\figpart=2}{

% Points
\node[fevertex] (v1) at (p1) {};
\node[fevertex-label, below left, anchor=north east] at (v1) {p1};

\node[fevertex] (v2) at (p2) {};
\node[fevertex-label, below right, anchor=north west] at (v2) {p2};

\node[fevertex] (v3) at (p3) {};
\node[fevertex-label, above right, anchor=south west] at (v3) {p3};

\node[fevertex] (v4) at (p4) {};
\node[fevertex-label, above left, anchor=south east] at (v4) {p4};

\node[fevertex] (v5) at (p5) {};
\node[fevertex-label, below, anchor=north] at (v5) {p5};

\node[fevertex] (v6) at (p6) {};
\node[fevertex-label, above, anchor=south] at (v6) {p6};

% Curves
\node (yneg1) at ($(p1)+(0.25*\domainx,0)$) {};
\node[curve-label, below] at (yneg1) {c\_yneg1};
\draw[curve-dir] ($(yneg1)+(-0.5*\dirsize,4)$) -- ++(\dirsize,0);

\node (yneg2) at ($(p5)+(0.25*\domainx,0)$) {};
\node[curve-label, below] at (yneg2) {c\_yneg2};
\draw[curve-dir] ($(yneg2)+(-0.5*\dirsize,4)$) -- ++(\dirsize,0);

\node (xpos) at ($(p2)+(0,0.5*\domainy)$) {};
\node[curve-label, right, rotate=90, anchor=north] at (xpos) {c\_xpos};
\draw[curve-dir] ($(xpos)+(-4,-0.5*\dirsize)$) -- ++(0,\dirsize);

\node (ypos2) at ($(p6)+(0.25*\domainx,0)$) {};
\node[curve-label, above] at (ypos2) {c\_ypos2};
\draw[curve-dir] ($(ypos2)+(+0.5*\dirsize,-4)$) -- ++(-\dirsize,0);

\node (ypos1) at ($(p4)+(0.25*\domainx,0)$) {};
\node[curve-label, above] at (ypos1) {c\_ypos1};
\draw[curve-dir] ($(ypos1)+(+0.5*\dirsize,-4)$) -- ++(-\dirsize,0);

\node (xneg) at ($(p1)+(0,0.5*\domainy)$) {};
\node[curve-label, left, rotate=-90, anchor=north] at (xneg) {c\_xneg};
\draw[curve-dir] ($(xneg)+(4,+0.5*\dirsize)$) -- ++(0,-\dirsize);

\node (fault) at ($(p5)+(0,0.5*\domainy)$) {};
\node[curve-label, right, rotate=90, anchor=north] at (fault) {c\_fault};
\draw[curve-dir] ($(fault)+(-4,-0.5*\dirsize)$) -- ++(0,\dirsize);


% Axes
\draw[axes,arrow] ($(p1)+(0,-\axesshift)$) to ($(p2)+(1.0*\axesshift,-\axesshift)$);
\node[axes-label,right] at ($(p2)+(1.0*\axesshift,-\axesshift)$) {x};
\foreach \ix in {0,1}{%
    \pgfmathsetmacro{\x}{-0.5*\domainx+\ix*\domainx}
    \draw[axes] ($(p1)+(\ix*\domainx,-\axesshift)$) -- ++(0,-\ticksize) node[below]{\x\,km};
}

\draw[axes,arrow] ($(p1)+(-\axesshift,0)$) to ($(p4)+(-\axesshift,+1.0*\axesshift)$);
\node[axes-label,above] at ($(p4)+(-\axesshift,1.0*\axesshift)$) {y};
\foreach \iy in {0,1}{%
    \pgfmathsetmacro{\y}{-0.5*\domainy+\iy*\domainy}
    \draw[axes] ($(p1)+(-\axesshift,\iy*\domainy)$) -- ++(-\ticksize,0) node[left]{\y\,km};
}

}{} % if/else


% -------------------------------------------------------------------------------------------------
% Gmsh Geometry
% -------------------------------------------------------------------------------------------------
\ifthenelse{\figpart=3}{

% Points
\node[fevertex] (v1) at (p1) {};
\node[fevertex-label, below left, anchor=north,xshift=3mm] at (v1) {v\_xneg\_yneg};

\node[fevertex] (v2) at (p2) {};
\node[fevertex-label, below right, anchor=north west] at (v2) {v\_xpos\_yneg};

\node[fevertex] (v3) at (p3) {};
\node[fevertex-label, above right, anchor=south west] at (v3) {v\_xpos\_yneg};

\node[fevertex] (v4) at (p4) {};
\node[fevertex-label, above left, anchor=south, xshift=3mm] at (v4) {v\_xneg\_ypos};

\node[fevertex] (v5) at (p5) {};
\node[fevertex-label, below, anchor=north] at (v5) {v\_fault\_yneg};

\node[fevertex] (v6) at (p6) {};
\node[fevertex-label, above, anchor=south] at (v6) {v\_fault\_ypos};

% Curves
\node (yneg1) at ($(p1)+(0.25*\domainx,0)$) {};
\node[curve-label, above] at (yneg1) {c\_yneg\_xneg};

\node (yneg2) at ($(p5)+(0.25*\domainx,0)$) {};
\node[curve-label, above] at (yneg2) {c\_yneg\_xpos};

\node (xpos) at ($(p2)+(0,0.5*\domainy)$) {};
\node[curve-label, right, rotate=90, anchor=north] at (xpos) {c\_xpos};

\node (ypos2) at ($(p6)+(0.25*\domainx,0)$) {};
\node[curve-label, below] at (ypos2) {c\_ypos\_xpos};

\node (ypos1) at ($(p4)+(0.25*\domainx,0)$) {};
\node[curve-label, below] at (ypos1) {c\_ypos\_xneg};

\node (xneg) at ($(p1)+(0,0.5*\domainy)$) {};
\node[curve-label, left, rotate=90, anchor=north] at (xneg) {c\_xneg};

\node (fault) at ($(p5)+(0,0.5*\domainy)$) {};
\node[curve-label, right, rotate=90, anchor=north] at (fault) {c\_fault};


% Axes
\draw[axes,arrow] ($(p1)+(0,-\axesshift)$) to ($(p2)+(1.0*\axesshift,-\axesshift)$);
\node[axes-label,right] at ($(p2)+(1.0*\axesshift,-\axesshift)$) {x};
\foreach \ix in {0,1}{%
    \pgfmathsetmacro{\x}{-0.5*\domainx+\ix*\domainx}
    \draw[axes] ($(p1)+(\ix*\domainx,-\axesshift)$) -- ++(0,-\ticksize) node[below]{\x\,km};
}

\draw[axes,arrow] ($(p1)+(-\axesshift,0)$) to ($(p4)+(-\axesshift,+1.0*\axesshift)$);
\node[axes-label,above] at ($(p4)+(-\axesshift,1.0*\axesshift)$) {y};
\foreach \iy in {0,1}{%
    \pgfmathsetmacro{\y}{-0.5*\domainy+\iy*\domainy}
    \draw[axes] ($(p1)+(-\axesshift,\iy*\domainy)$) -- ++(-\ticksize,0) node[left]{\y\,km};
}

}{} % if/else


% -------------------------------------------------------------------------------------------------
% Step 1
% -------------------------------------------------------------------------------------------------
\ifthenelse{\figpart=4}{

    % Dirichlet BC on +x and -x boundaries
    \node[ground, rotate=-90, minimum width=75mm] (ground) at ($(p1)+(0,0.5*\domainy)$) {};
    \node[bc-label, left] at (ground.south) {$\begin{array}{l} u_x=0 \\ u_y=0 \end{array}$};

    \node[ground, rotate=90, minimum width=75mm] (ground) at ($(p2)+(0,0.5*\domainy)$) {};
    \node[bc-label, right] at (ground.south) {$\begin{array}{l} u_x=0 \\ u_y=0 \end{array}$};

    % Fault slip
    \coordinate (x) at ($(p5)+(0,0.5*\domainy)$);
    \draw[bc,arrow] ($(x)+(+4,+0.5*\bndryshift)$) -- ++(0,-\bndryshift);
    \draw[bc,arrow] ($(x)+(-4,-0.5*\bndryshift)$) -- ++(0,\bndryshift);
    \node[bc-label, right, xshift=5] at (x) {$d=-2.0\textrm{m}$};

}{} % if/else


% -------------------------------------------------------------------------------------------------
% Step 2
% -------------------------------------------------------------------------------------------------
\ifthenelse{\figpart=5}{

    % Dirichlet BC on +x and -x boundaries
    \coordinate (x) at ($(p1)+(0,0.5*\domainy)$);
    \draw[bc,arrow] ($(x)+(0,-0.5*\bndryshift)$) -- ++(0,+\bndryshift) node[midway,anchor=east,bc-label] {$\begin{array}{l}u_x=0 \\ \dot{u}_y=+1.0\,\textrm{cm/yr} \end{array}$};

    \coordinate (x) at ($(p2)+(0,0.5*\domainy)$);
    \draw[bc,arrow] ($(x)+(0,+0.5*\bndryshift)$) -- ++(0,-\bndryshift) node[midway,anchor=west,bc-label] {$\begin{array}{l}u_x=0 \\ \dot{u}_y=-1.0\,\textrm{cm/yr} \end{array}$};

    % Fault slip
    \coordinate (x) at ($(p5)+(0,0.5*\domainy)$);
    \draw[bc,arrow] ($(x)+(+4,+0.5*\bndryshift)$) -- ++(0,-\bndryshift);
    \draw[bc,arrow] ($(x)+(-4,-0.5*\bndryshift)$) -- ++(0,\bndryshift);
    \node[bc-label, right, xshift=5] at (x) {$\begin{array}{l}d=-2.0 \textrm{m} \\ t\ge100 \textrm{yr}\end{array}$};

}{} % if/else


% -------------------------------------------------------------------------------------------------
% Step 3
% -------------------------------------------------------------------------------------------------
\ifthenelse{\figpart=6}{

    % Dirichlet BC on +x and -x boundaries
    \coordinate (x) at ($(p1)+(0,0.5*\domainy)$);
    \draw[bc,arrow] ($(x)+(0,-0.5*\bndryshift)$) -- ++(0,+\bndryshift) node[midway,anchor=east,bc-label] {$\begin{array}{l}u_x=0 \\ \dot{u}_y=+1.0\,\textrm{cm/yr} \end{array}$};

    \coordinate (x) at ($(p2)+(0,0.5*\domainy)$);
    \draw[bc,arrow] ($(x)+(0,+0.5*\bndryshift)$) -- ++(0,-\bndryshift) node[midway,anchor=west,bc-label] {$\begin{array}{l}u_x=0 \\ \dot{u}_y=-1.0\,\textrm{cm/yr} \end{array}$};

    % Fault slip
    \coordinate (x) at ($(p5)+(0,0.5*\domainy)$);
    \draw[bc,arrow] ($(x)+(+4,+0.5*\bndryshift)$) -- ++(0,-\bndryshift);
    \draw[bc,arrow] ($(x)+(-4,-0.5*\bndryshift)$) -- ++(0,\bndryshift);
    \node[bc-label, right, xshift=5] at (x) {$\begin{array}{l}d=-1.0 \textrm{m} \\ t\ge100 \textrm{yr} \\[3mm] d=-3.0 \textrm{m} \\ t\ge200 \textrm{yr}\end{array}$};

}{} % if/else


% -------------------------------------------------------------------------------------------------
% Step 4
% -------------------------------------------------------------------------------------------------
\ifthenelse{\figpart=7}{

    % Dirichlet BC on +x and -x boundaries
    \node[ground, rotate=-90, minimum width=75mm] (ground) at ($(p1)+(0,0.5*\domainy)$) {};
    \node[bc-label, left] at (ground.south) {$\begin{array}{l} u_x=0 \\ u_y=0 \end{array}$};

    \node[ground, rotate=90, minimum width=75mm] (ground) at ($(p2)+(0,0.5*\domainy)$) {};
    \node[bc-label, right] at (ground.south) {$\begin{array}{l} u_x=0 \\ u_y=0 \end{array}$};

    % Fault slip
    \coordinate (x) at ($(p5)+(0,0.5*\domainy)$);
    \draw[bc,arrow] ($(x)+(+4,-0.5*\bndryshift)$) -- ++(0,+\bndryshift);
    \draw[bc,arrow] ($(x)+(-4,+0.5*\bndryshift)$) -- ++(0,-\bndryshift);
    \node[bc-label, right, xshift=5] at (x) {$d=d(y)$};

}{} % if/else


% -------------------------------------------------------------------------------------------------
% Step 5
% -------------------------------------------------------------------------------------------------
\ifthenelse{\figpart=8}{

    % Dirichlet BC on +x and -x boundaries
    \node[ground, rotate=-90, minimum width=75mm] (ground) at ($(p1)+(0,0.5*\domainy)$) {};
    \node[bc-label, left] at (ground.south) {$\begin{array}{l} u_x=0 \\ u_y=0 \end{array}$};

    \node[ground, rotate=90, minimum width=75mm] (ground) at ($(p2)+(0,0.5*\domainy)$) {};
    \node[bc-label, right] at (ground.south) {$\begin{array}{l} u_x=0 \\ u_y=0 \end{array}$};

    % Fault slip
    \coordinate (x) at ($(p5)+(0,0.5*\domainy)$);
    \draw[bc,arrow] ($(x)+(+4,-0.5*\bndryshift)$) -- ++(0,+\bndryshift);
    \draw[bc,arrow] ($(x)+(-4,+0.5*\bndryshift)$) -- ++(0,-\bndryshift);
    \node[bc-label, right, xshift=5] at (x) {$d=\delta(y)$};

}{} % if/else


\end{tikzpicture}}

\end{document}
